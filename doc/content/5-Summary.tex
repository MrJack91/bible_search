\chapter{Schlussfolgerung}

\section{Erkenntnisse}
Apache Lucene stellt ein mächtiges Java-Framework um eine Suchmaschine zu Implementieren zur Verfügung.
Es werden viele Optimierungspotenziale angeboten. Gewichtung, Suggest-Feature, Synonyme, Highlighting der Resultate und viele mehr steht gratis zur Verfügung.
Trotzdem kann eine einfache Suche in kurzer Zeit umgesetzt werden.

Apache Lucene könnte mit viel mehr Dokumenten als nur die der Bibel benutzt werden. Aber auch für reduzierte Inhalte hat man eine schnelle Suche mit all den erwähnten Features.

Um eine nützliche Suche umzusetzen ist eine gute Planung Voraussetzung. Die Bewertung ist sehr zeit intensiv und muss sinnvoll und viel aussagend gewählt werden.

\section{Mögliche Erweiterungen}
Es gibt noch viele interessante Funktionen für die Suche durchaus sinnvoll und interessant sind.

Besonders nützlich wäre eine \textit{Synonym Expansion}, dabei werden nebst den eingegebenen Begriffen auch Synonyme gefunden. Gerade die Bibel die oft in einer etwas älterer Sprache vorliegt, könnte man sich durch Synonym Expansion einen Mehrwert versprechen.
Das Hauptproblem ist, dass die Synonyme definiert sein müssen. Für die deutsche Sprache wurde keine freie, im solche Bibliothek im passenden Format gefunden. Manuell eine Bibliothek zu erstellen lohnt sich in Anbetracht der dazu notwendigen Zeit und der Textmenge nicht.

Um die Suchmaschine und die Precision zu erhöhen, sollten \glspl{glos:documentLabel} die nur aufgrund des Kontextes erscheinen, von den Ergebnissen ausgeschlossen werden. Für den Nutzer ist sonnst nicht ersichtlich, wieso der Vers gefunden wird.

Weitere Erweiterungsmöglichkeiten auf die nur oberflächlich eingegangen wird sind:
\begin{itemize}[noitemsep]
	\item \textbf{Mehrwortgruppenidentifikation:} Zusammengesetzte Wörter als Einzel-Wörter betrachten, dabei aber den Kontext trotzdem beachten.
	
	\item \textbf{Terminologische Kontrolle:} Antonyme, verwandte Wörter, Oberbegriffe, speziellere Begriffe ebenfalls berücksichtigen.

	\item \textbf{Weitere Bibelübersetzungen:} Anstatt nur die Luther Übersetzung sollten mehrere und auch modernere Übersetzungen angeboten werden. Diese Erweiterung bringt viele notwendigen Anpassungen mit sich: Aufbau des Indexes, Suchmaske und Auflistung der Resultate müsste stark überarbeitet werden.

	\item \textbf{Suche nach konkreten Bibelstellen:} Nebst der Suche nach dem Inhalt von Versen, ist das aufschlagen von bekannten Versen (über die Stellenangabe) sehr nützlich.
\end{itemize}


\section{Ausblick}
Diese Arbeit endet mit der Abgabe des Dokumentes.
Es ist keine Weiterentwicklung der Suchmaschine geplant. Der Code ist öffentlich verfügbar und darf als Grundlage oder als Beispielcode weiter verwendet und weiterentwickelt werden.


\section{Persönliches Schlusswort}
Persönlich war es ein sehr spannender Einblick ins Thema Information Retrieval.

Dadurch, dass ich die Aufgabe grösstenteils selbst definieren durfte, konnte ich viele meiner Vorstellungen umsetzen.
Durch die Zusammenführung der zwei Themen \textit{Information Retrieval} und \textit{die Bibel}, konnte ich mich gleich mit zwei interessanten Gebieten beschäftigen.

Da ich mich bisher nicht in dieser Tiefe mit Suchmaschinen befasst habe, gewann ich einen grossen Einblick in dieses Thema. Dies führte zu vielen neue Überlegungen, Gedanken und Erkenntnisse angefangen bei der Funktionsweise von Indexen bis hin zur Bewertung und Kennzahlen einer Suche.

Mir hat diese Arbeit sehr zugesagt.
