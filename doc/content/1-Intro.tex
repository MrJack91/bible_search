\chapter{Einleitung}
Im Rahmen einer Seminararbeit zum Thema Information Retrieval an der \gls{zhawLabel} wurde eine Volltext-Suche über die \gls{glos:bibleLabel} umgesetzt. Die Suche basiert auf Apache Lucene.\footcite{Apache_Lucene_2016-05-07}

\section{Ausgangslage}
Die Bibel wird auf verschiedenen Medien in der ganzen Welt günstig, wenn nicht sogar gratis verteilt.
Seit es Smartphones gibt, werden auch Apps angeboten die einem das Buch jederzeit zur Verfügung stellen.

Viele der elektronischen Ausführungen unterstützen eine Volltextsuche (simples Text-Matching).
Will man aber eine bestimmte Stelle finden und weiss nur noch den ungefähren Inhalt, so versagt die Volltextsuche oft.

\subsection{Motivation}
Mit Hilfe dieser Arbeit soll die Suche nach gewünschten Passagen verbessert werden, indem die Vorteile einer gängigen Text Search Engine verwendet werden.
So soll ein nützliches Werkzeug für \gls{glos:bibleLabel}-Interessierte angeboten werden.

\subsection{Inhalt}
Nebst der Umsetzung der eigentlichen Suche, wird die Suche mit anderen, bereits bestehenden Suchen verglichen, so dass Veränderungen festgestellt werden können.

\subsection{Abgrenzungen}
Die umzusetzende Suche soll nicht die Funktion eines \gls{glos:bibleReaderLabel}s anbieten und somit den Gebrauch der \gls{glos:bibleLabel} ersetzen, sondern es soll lediglich ein Werkzeug sein, dass einem Referenzen mit minimalem Kontext zu Bibelstellen ausgibt, die dann in \glspl{glos:bibleReaderLabel} nachgeschlagen werden können.

\section{Exkurs: Bibelstruktur}
Für nicht \gls{glos:bibleLabel}-Kenner wird folgend der Aufbau und die Struktur der \gls{glos:bibleLabel} beschrieben.

Die \gls{glos:bibleLabel} gilt als Sammlung von 66 verschiedenen Büchern.
Sie ist aufgeteilt in zwei Testamente:
\begin{itemize}
	\item \gls{atLabel} mit 39 Bücher
	\item \gls{ntLabel} mit 27 Bücher
\end{itemize}
Jedes Buch ist in Kapitel und Verse eingeteilt.

Es gibt verschiedene Übersetzungen (Formulierungen), welche den Text eindeutig identifizieren.

%todo: Anhang mit vollständiger Liste der Bücher und Abkürzungen

\begin{framed}
	\textbf{Beispiel: \gls{glos:bibleLabel}-Vers mit vollständiger Stellenangabe}
	\begin{quote}
		"`Denn also hat Gott die Welt geliebt, dass er seinen eingeborenen Sohn gab, damit alle, die an ihn glauben, nicht verloren werden, sondern das ewige Leben haben."' - \textit{Joh 3,16 (MLB)\footnotemark}
	\end{quote}
\end{framed}
\footnotetext{Angabe der genauen Übersetzungen: MLB = Martin Luther Bibel}
