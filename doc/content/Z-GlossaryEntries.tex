% !TeX encoding=utf8
% !TeX spellcheck = de_CH_frami

%%% --- Acronym definitions
\IfDefined{newacronym}{%
% examples

% our used acronyms
\newacronym{apiLabel}{API}{Application Programming Interface}
\newacronym{atLabel}{AT}{Altes Testament}
\newacronym{cliLabel}{CLI}{Command Line Interface}
\newacronym{guiLabel}{GUI}{Graphical User Interface}
\newacronym{ntLabel}{NT}{Neues Testament}
\newacronym{mlbLabel}{MLB}{Martin Luther Bibel}
\newacronym{zhawLabel}{ZHAW}{Zürcher Hochschule für Angewandte Wissenschaften}


}

%%% --- Symbol list entries

%\newglossaryentry{symb:Pi}{%
%  name=$\pi$,%
%  description={mathematical constant},%
%  sort=symbolpi, type=symbolslist%
%}


%%% --- Glossary entries
\newglossaryentry{glos:bibleLabel}{
	name=Bibel,
	plural={Bibel},
	description={Als Bibel (altgr. biblia für „Bücher“; daher auch \textit{Buch der Bücher}) bezeichnet man eine Schriftensammlung, die im Judentum und Christentum als Heilige Schrift mit normativem Anspruch für die ganze Religionsausübung gilt.\footcite{Bibel_Wikipedia_2016-05-07}. Die Bibel 66 Bücher mit 1189 Kapiteln und mehr als 31.100 Versen.}
}

\newglossaryentry{glos:bibleReaderLabel}{
	name=Bibel-Reader,
	plural={Bibel-Readern},
	description={Mit Bibel-Readern wird alles bezeichnet, was die Möglichkeit bietet in der ganzen \gls{glos:bibleLabel} zu lesen und gewünschte Stellen nachzuschlagen. Dies kann in gedruckter oder digitaler Form sein.}
}





% use it with \gls{glos:DVD}
% use plural with \glspl{thinClientLabel}
