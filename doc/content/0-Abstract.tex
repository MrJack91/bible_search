% !TeX encoding=utf8
% !TeX spellcheck = de_CH_frami

\subsection*{Kurzverfassung}
Die vorliegende Arbeit zeigt die Implementation einer Bibeltext-Suchmaschine mit Apache Lucene 6.0.
Im konzeptionellen Teil wird die am Besten passende Dokumenteinheit definiert.
Die Suchmaschine wird anhand Beispiel-Anfragen mit anderen externen Suchmaschinen über bekannte Kennzahlen verglichen.

\begin{flushright}
	\textit{Michael Hadorn}	
\end{flushright}

\vfill

%
\mbox{}\\[0.5\baselineskip]\noindent
\textbf{Schlagwörter:} 
Apache Lucene 6.0, Bibel search engine, Bibel Suchmaschine, Information Retrieval

\newpage
\subsection*{Versionsübersicht}
\begin{center}
	\centering
	\small\renewcommand{\arraystretch}{1.4}
	\rowcolors{1}{tablerowcolor}{tablebodycolor}
	\begin{tabularx}{1.0\textwidth}{ R{0.1\linewidth} | R{0.16\linewidth} | X  }%
		\hline
		\textbf{Version*} & \textbf{Datum} & \textbf{Kommentar}\\
		\hline\hline
		v0.0.1 & 28.04.16 & Dokument Erstellung \\
		v0.1.0 & 11.05.16 & Abgabe des Zwischenstandes \\
		v1.0.0 & 13.05.16 & Finale Abgabe \\
		\hline\hline
	\end{tabularx}
\end{center}
\vspace{-1.0\baselineskip}
{\footnotesize * Eine genauere Übersicht der Änderungen kann auf GitHub eingesehen werden.\footcite{github_bibleSearch_2016-05-07}}

\vfill

\subsection*{Danksagung}
Mein besonderer Dank gilt Karin Hadorn für die vollständige und detaillierte Korrekturlesung und das geteilte Interesse an den durch diese Arbeit gewonnenen Erkenntnissen von mir.\\
