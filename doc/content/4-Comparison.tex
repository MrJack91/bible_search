\chapter{Vergleich zu anderen Suchmaschinen}
 \label{sec:compareSearches}
Um einen Vergleich aufzustellen, werden Beispielanfragen mit der umgesetzten Suche durchgeführt und analysiert.
Anschliessend werden die Beispielanfragen auf bestehenden Suchen durchgeführt.

\section{Mit der umgesetzten Suche}
Für die Suche wird mindestens ein Teil der Resultat-Menge aufgelistet. Falsche Resultate werden durchgestrichen dargestellt. Mit gelb werden die explizit verlangten Resultate hervorgehoben.
Bei den Resultaten die weder gelb noch durchgestrichen sind, stimmt der Context, aber sie waren nicht explizit verlangt.

Um schlussendlich einen sinnvollen Wert für Precision und Recall zu erhalten, müssen die Ergebnisse noch gruppiert werden.
Besonders wenn als Ziel einen Bereich von Versen vorgegeben ist.
Da die Suche an sich vers-/satzbasiert umgesetzt worden ist, müssen die Ergebnisse entsprechend manuell zu den vorgegebenen Versbereichen gruppiert werden.

Ersichtlich wird dies in der Auswertung im \cref{subsec:index_abendmahl}.


\subsection{Suche zum Thema Abendmahl}
\begin{table}[H]
	\centering
	\small\renewcommand{\arraystretch}{1.4}
	\rowcolors{1}{tablerowcolor}{tablebodycolor}
	%
	\captionabove{Query: Vorkommen des Abendmahles}
	%
	\begin{tabularx}{0.9\textwidth}{ L{0.15\linewidth} | X  }%
		\hline
		Fragestellung: & \textit{Wo wurde das Abendmahl eingeführt?} und \textit{Wo wird es erwähnt?}\\
		Anfrage-Typ: & \textit{Vers orientierte Anfrage}\\
		Beschreibung: & Der Nutzer will wissen, wo das Abendmahl erwähnt wird.\\
		Query: & "`brot AND leib"'\\
		Erwartung: & 
		Vers und Stellenangabe:
		\begin{itemize}[noitemsep]
			\item Mt 26, 26-28 \gls{lutLabel} ($\rightarrow$ Kontext: 17–29)
			\item Mk 14, 22 \gls{lutLabel} ($\rightarrow$ Kontext: 12-26)
			\item Lk 22, 17-20 \gls{lutLabel} ($\rightarrow$ Kontext: 14–20)
			\item Joh 13,2–4 \gls{lutLabel}
			\item 1 Kor 11, 24 \gls{lutLabel} ($\rightarrow$ Kontext: 23–27)
		\end{itemize}\\
		\hline
	\end{tabularx}
\end{table}

\label{subsec:index_abendmahl}
Die Suche nach dem Abendmahl (aus \cref{tab:query_abendmahl}), ergibt folgende Ergebnisse:\\
18 total matching documents
\begin{itemize}[noitemsep]
	\item 1.	\st{Denn ein Brot ist's, so sind wir viele ein Leib, dieweil wir alle eines Brotes teilhaftig sind. - 1 Kor 10,17}
	\item 2.	Der gesegnete Kelch, welchen wir segnen, ist der nicht die Gemeinschaft des Blutes Christi? Das Brot, das wir brechen, ist das nicht die Gemeinschaft des Leibes Christi? - 1 Kor 10,16
	\item 3.	Welcher nun unwürdig von diesem Brot isset oder von dem Kelch des HERRN trinket, der ist schuldig an dem Leib und Blut des HERRN. - \hl{1 Kor 11,27}
	\item 4.	Und er nahm das Brot, dankte und brach's und gab's ihnen und sprach: Das ist mein Leib, der für euch gegeben wird; das tut zu meinem Gedächtnis. - \hl{Lk 22,19}
	\item 5.	Da sie aber aßen, nahm Jesus das Brot, dankte und brach's und gab's den Jüngern und sprach: Nehmet, esset; das ist mein Leib. - \hl{Mt 26,26}
	\item 6.	Und indem sie aßen, nahm Jesus das Brot, dankte und brach's und gab's ihnen und sprach: Nehmet, esset; das ist mein Leib. - \hl{Mk 14,22}
	\item 7.	Denn so oft ihr von diesem Brot esset und von diesem Kelch trinket, sollt ihr des HERRN Tod verkündigen, bis daß er kommt. - \hl{1 Kor 11,26}
	\item 8.	Der Mensch prüfe aber sich selbst, und also esse er von diesem Brot und trinke von diesem Kelch. - \hl{1 Kor 11,28}
	\item 9.	\st{Da brachten sie Joseph ihr Vieh; und er gab ihnen Brot um ihre Pferde, Schafe, Rinder und Esel. Also ernährte er sie mit Brot das Jahr um all ihr Vieh. - 1 Mo (Gen) 47,17}
	\item 10.	\st{Ich rief meine Freunde an, aber sie haben mich betrogen. Meine Priester und Ältesten in der Stadt sind verschmachtet; denn sie gehen nach Brot, damit sie ihre Seele laben. - Kla 1,19}
	\item 11.	\st{Sie sollen auch keine Platte machen auf ihrem Haupt noch ihren Bart abscheren und an ihrem Leib kein Mal stechen. - 3 Mo (Lev) 21,5}
	\item 12.	Ich habe es von dem HERRN empfangen, das ich euch gegeben habe. Denn der HERR Jesus in der Nacht, da er verraten ward, nahm das Brot,
	dankte und brach's und sprach: Nehmet, esset, das ist mein Leib, der für euch gebrochen wird; solches tut zu meinem Gedächtnis. - \hl{1 Kor 11,23-24}
	\item 13.	\st{... nichts mehr übrig vor unserm Herrn denn unsre Leiber und unser Feld. - 1 Mo (Gen) 47,18}
	\item 14.	\st{Ich habe schier meine Augen ausgeweint, daß mir mein Leib davon wehe tut; ...
	sprachen: Wo ist Brot und Wein? ... - Kla 2,11-12}
	\item 15.	Da antwortete Judas, der ihn verriet, und sprach: Bin ich's Rabbi? Er sprach zu ihm: Du sagst es. - \hl{Mt 26,25}
	\item 16.	denn ich sage euch: Ich werde nicht trinken von dem Gewächs des Weinstocks, bis das Reich Gottes komme. - \hl{Lk 22,18}
	\item 17.	\st{Als mit den Klugen rede ich; richtet ihr, was ich sage. - 1 Kor 10,15}
	\item 18.	Zwar des Menschen Sohn geht hin, wie von ihm geschrieben steht; weh aber dem Menschen, durch welchen des Menschen Sohn verraten wird. Es wäre demselben Menschen besser, daß er nie geboren wäre. - \hl{Mk 14,21}
\end{itemize}

\todo{erklären wie die Zahlen gruppiert worden sind.}
Recall und Precision lauten wie folgt:
\begin{table}[H]
	\centering
	\small\renewcommand{\arraystretch}{1.4}
	\rowcolors{1}{tablerowcolor}{tablebodycolor}
	\captionabove{Suche zum Thema Abendmahl: Precision und Recall}
	\label{tab:index_abendmahl}
	\begin{tabularx}{0.7\textwidth}{ L{0.22\linewidth} | L{0.1\linewidth} | X }%
		\hline
		 & Anzahl & Erklärung \\ \hline \hline
		Relevant: & 5 & \\
		Relevant \& gefunden: & 4 & Gruppen: 2+2+2+0+4 = 10\\
		Fehler: & 7 & \\
		Ignorierte Resultate: & 1 & \\
		Total Ergebnisse: & 18 & \\
		Total korrigiert: & 11 & \\
		\hline
		\textbf{Recall:} & \textbf{0.8} & $= \frac{4}{5}$\\
		\textbf{Precision:} & \textbf{0.36} & $= \frac{4}{11}$ \\
		\hline\hline
	\end{tabularx}
\end{table}



\subsection{Suche zum Sohn von Abraham}
\begin{table}[H]
	\centering
	\small\renewcommand{\arraystretch}{1.4}
	\rowcolors{1}{tablerowcolor}{tablebodycolor}
	%
	\captionabove{Query: Sohn von Abraham}
	\label{tab:query_abendmahl}
	%
	\begin{tabularx}{0.9\textwidth}{ L{0.15\linewidth} | X  }%
		\hline
		Frage: & \textit{Wie hiess der Sohn vom Abraham?}\\
		Anfrage-Typ: & \textit{Fakten orientierte Anfrage}\\
		Beschreibung: & Der Nutzer will den Namen des Sohnes von Abraham wissen.\\
		Query: & - \\
		Erwartung: & Vers und Stellenangabe:
		"`Dies ist das Geschlecht Isaaks, des Sohnes Abrahams: Abraham zeugte Isaak."' - \textit{1 Mo (Gen) 25,19 \gls{lutLabel}}\\
		\hline
	\end{tabularx}
\end{table}











\begin{table}[H]
	\centering
	\small\renewcommand{\arraystretch}{1.4}
	\rowcolors{1}{tablerowcolor}{tablebodycolor}
	%
	\captionabove{Query 2: "`Denn so sehr hat Gott die Welt geliebt, ..."'}
	%
	\begin{tabularx}{0.9\textwidth}{ L{0.15\linewidth} | X  }%
		\hline
		Frage: & \textit{Wie heisst der bekannte Vers ähnlich: "`Denn so sehr hat Gott die Welt geliebt, dass..."'}\\
		Anfrage-Typ: & \textit{Vers orientierte Anfrage}\\
		Beschreibung: & Der Nutzer will den genauen Vers sehen inkl. Stellenangabe\\
		Query: & -\\
		Erwartung: & Vers und Stellenangabe:
		"`Also hat Gott die Welt geliebt, dass er seinen eingeborenen Sohn gab"' - \textit{Joh 3,16 \gls{lutLabel}}\\
		\hline
	\end{tabularx}
\end{table}



\begin{table}[H]
	\centering
	\small\renewcommand{\arraystretch}{1.4}
	\rowcolors{1}{tablerowcolor}{tablebodycolor}
	%
	\captionabove{Query 4: \textit{10 Gebote}}
	%
	\begin{tabularx}{0.9\textwidth}{ L{0.15\linewidth} | X  }%
		\hline
		Frage: & \textit{Wo stehen die 10 Gebote?} und \textit{Wie lauten sie?}\\
		Anfrage-Typ: & \textit{Vers und Fakten orientierte Anfrage}\\
		Beschreibung: & Der Nutzer will sich über die 10 Gebote informieren.\\
		Query: & -\\
		Erwartung: & Vers und Stellenangabe:\\
		\hline
	\end{tabularx}
\end{table}

