\chapter{Umsetzung der Suche}
In diesem Kapitel wird die Umsetzung der Suche beschrieben.

Die Suche soll mit Apache Lucene umgesetzt werden. Für die Bedienun genügt in erster Linie ein \gls{cliLabel}.

\section{Beschreibung der Suchresultaten}
Für jedes Suchresultat soll der Text und die genaue Bibelstelle im Format "`[Buch Abk.], [Kapitel],[Vers]"' ausgeben werden.
Die Resultate sollen nach einem sinnvollen Ranking sortiert sein.
Besonders schön wäre eine Farbliche Hervorhebung (Highlighting) der Suchbegriffe in der Resultaten-Liste.

\textbf{Beispiel der Resultat-Liste Ergebnisses:}
\begin{itemize}
	\item Da wurden sie alle gutes Muts und nahmen auch Speise. - Apg 27,36
	\item Was setzt sich dein Mut gegen Gott, daß du solche Reden aus deinem Munde lässest? - Hiob 15,13
	\item der den Fürsten den Mut nimmt und schrecklich ist unter den Königen auf Erden. - Ps 76,12
	\item Ein Geduldiger ist besser denn ein Starker, und der seines Mutes Herr ist, denn der Städte gewinnt. - Spr 16,32
\end{itemize}


\section{Optionale Dokumenteinheit}


\section{Indexierung}
\subsection{Besonderheiten}


\section{Suche}
\subsection{Anfrage-Syntax}

\section{Features}

* German Context
* Ranking
* stop words filtering
* stemming
* text normalization
* suggest
* highlighter
* (ausstehend) - synonym expansion

\section{Index Analysierung mit Luke}
