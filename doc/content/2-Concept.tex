\chapter{Konzept}
In diesem Kapitel wird das Konzept für die Messkriterien und die zu verwendenden Beispielanfragen beschrieben.

\section{Vorgehen}
Um den Nutzern einen echten Mehrwert einer weiteren Bibelsuche anbieten zu können, müssen existierende Suchen analysiert und bewertet werden.
Dazu sollen \textit{Kriterien für die Messbarkeit} einer Suche und deren Resultaten definiert werden, welche dann für jede Suche "`gemessen"' werden kann.
Weiter werden Beispielanfragen definiert, die in allen Suchen ausgeführt werden und so die Basis für den Vergleich mit der neuen Suche bilden.

\section{Kriterien für die Messbarkeit}
%Precision = (Relevant & Gefunden) / Gefunden
%	—> Genauigkeit
%Recall = (Relevant & Gefunden) / Relevant
%	—> Vollständigkeit

\section{Suchanfrage }

\subsection{Anfrage Typen}
Hauptsächlich gibt es zwei Arten von Suchanfragen die ein Nutzer stellen kann:
\begin{itemize}
	\item \textbf{Faktenorientierte Anfrage}\\
		Bei der faktenorientierten Anfrage steht der in der Bibel beschriebene Sachverhalt im Vordergrund. Der Nutzer will möglich rasch eine Antwort auf seine Frage "`Wie ist es in der Bibel beschrieben?"' finden.

	\item \textbf{Versorientierte Anfrage}\\
		Die versorientierten Anfrage soll dem Nutzer helfen Versen zu gewünschten Themen finden zu können.
		Zudem kann der Nutzer ebenfalls die vollständigen Verse finden, wenn ihm nur noch der ungefähren Inhalt bekannt ist.
\end{itemize}


\subsection{Beispielanfragen}
\label{subsec:exampleQueries}

\begin{table}[H]
	\centering
	\small\renewcommand{\arraystretch}{1.4}
	\rowcolors{1}{tablerowcolor}{tablebodycolor}
	%
	\captionabove{Q1: Sohn von Abraham}
	%
	\begin{tabularx}{0.9\textwidth}{ L{0.15\linewidth} | X  }%
		\hline
		Frage: & \textit{Wie hiess der Sohn vom Abraham?}\\
		Anfrage-Typ: & \textit{Faktenorientierte Anfrage}\\
		Beschreibung: & Der Nutzer will den Namen des Sohnes von Abraham wissen.\\
		Query: & - \\
		Erwartung: & Vers und Stellenangabe:
		"`Dies ist das Geschlecht Isaaks, des Sohnes Abrahams: Abraham zeugte Isaak."' - \textit{1 Mo (Gen) 25,19 \gls{mlbLabel}}\\
		\hline
	\end{tabularx}
\end{table}



\begin{table}[H]
	\centering
	\small\renewcommand{\arraystretch}{1.4}
	\rowcolors{1}{tablerowcolor}{tablebodycolor}
	%
	\captionabove{Q2: "`Denn so sehr hat Gott die Welt geliebt, ..."'}
	%
	\begin{tabularx}{0.9\textwidth}{ L{0.15\linewidth} | X  }%
		\hline
		Frage: & \textit{Wie heisst der bekannte Vers ähnlich: "`Denn so sehr hat Gott die Welt geliebt, dass..."'}\\
		Anfrage-Typ: & \textit{Versorientierte Anfrage}\\
		Beschreibung: & Der Nutzer will den genauen Vers sehen inkl. Stellenangabe\\
		Query: & -\\
		Erwartung: & Vers und Stellenangabe:
		"`Also hat Gott die Welt geliebt, dass er seinen eingeborenen Sohn gab"' - \textit{Joh 3,16 \gls{mlbLabel}}\\
		\hline
	\end{tabularx}
\end{table}

\begin{table}[H]
	\centering
	\small\renewcommand{\arraystretch}{1.4}
	\rowcolors{1}{tablerowcolor}{tablebodycolor}
	%
	\captionabove{Q3: Abendmahl Ersterwähnung}
	%
	\begin{tabularx}{0.9\textwidth}{ L{0.15\linewidth} | X  }%
		\hline
		Frage: & \textit{Wo wurde das Abendmahl hinzugefügt?}\\
		Anfrage-Typ: & \textit{Faktenorientierte Anfrage}\\
		Beschreibung: & Der Nutzer den Kontext vom Abendmahl finden.\\
		Query: & -\\
		Erwartung: & Vers und Stellenangabe: \\
		\hline
	\end{tabularx}
\end{table}

\begin{table}[H]
	\centering
	\small\renewcommand{\arraystretch}{1.4}
	\rowcolors{1}{tablerowcolor}{tablebodycolor}
	%
	\captionabove{Q4: \textit{10 Gebote}}
	%
	\begin{tabularx}{0.9\textwidth}{ L{0.15\linewidth} | X  }%
		\hline
		Frage: & \textit{Wo stehen die 10 Gebote?} und \textit{Wie lauten sie?}\\
		Anfrage-Typ: & \textit{Vers- und faktenorientierte Anfrage}\\
		Beschreibung: & Der Nutzer will sich über die 10 Gebote informieren.\\
		Query: & -\\
		Erwartung: & Vers und Stellenangabe:\\
		\hline
	\end{tabularx}
\end{table}


% Weitere Themen:
% - Wer sucht der wird finden.. -> math 7.7
% - Homosexualität
% - Der HERR segne dich und behüte dich; der HERR lasse sein Angesicht leuchten über dir und sei dir gnädig; der HERR hebe sein Angesicht über dich und gebe dir Frieden. - 4 Mose 6:24-26 | LUT |
% Nächsten Liebe

