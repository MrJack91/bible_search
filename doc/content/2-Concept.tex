\chapter{Konzept}
In diesem Kapitel wird das Konzept für die Messkriterien und die zu verwendenden Beispielanfragen beschrieben.

\section{Vorgehen}
Um den Nutzern einen echten Mehrwert einer weiteren Bibelsuche anbieten zu können, müssen existierende Suchen analysiert und bewertet werden.
Dazu sollen \textit{Kriterien für die Messbarkeit} einer Suche und deren Resultaten definiert werden, welche dann für jede Suche "`gemessen"' werden kann.
Weiter werden Beispielanfragen definiert, die in allen Suchen ausgeführt werden und so die Basis für den Vergleich mit der neuen Suche bilden.

\section{Kriterien für die Messbarkeit}
%Precision = (Relevant + Gefunden) / Gefunden
%	—> Genauigkeit
%Recall = (Relevant + Gefunden) / Relevant
%	—> Vollständigkeit


\section{Suchanfrage }

\subsection{Anfrage Typen}
Hauptsächlich gibt es zwei Arten von Suchanfragen die ein Nutzer stellen kann:
\begin{itemize}
	\item \textbf{Faktenorientierte Anfrage}\\
		Bei der faktenorientierten Anfrage steht der in der Bibel beschriebene Sachverhalt im Vordergrund. Der Nutzer will möglich rasch eine Antwort auf seine Frage "`Wie ist es in der Bibel beschrieben?"' finden.

	\item \textbf{Versorientierte Anfrage}\\
		Die versorientierten Anfrage soll dem Nutzer helfen Versen zu gewünschten Themen finden zu können.
		Zudem kann der Nutzer ebenfalls die vollständigen Verse finden, wenn ihm nur noch der ungefähren Inhalt bekannt ist.
\end{itemize}


\subsection{Beispielanfragen}
